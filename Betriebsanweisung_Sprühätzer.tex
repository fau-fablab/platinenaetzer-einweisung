\documentclass[fontsize=9pt]{scrartcl}
\pagestyle{empty}
\usepackage{lmodern}
\renewcommand*\familydefault{\sfdefault}
\usepackage[utf8]{inputenc}
\usepackage[T1]{fontenc}
\usepackage[ngerman]{babel}
\usepackage[babel,german=quotes]{csquotes}
\usepackage{graphicx}
\usepackage{xcolor}
\usepackage{geometry}
\geometry{a4paper, top=15mm, left=20mm, right=20mm, bottom=15mm, headsep=0mm, footskip=0mm}

\usepackage{tikz}
\makeatletter
\setkomafont{section}{\color{white}
    \bfseries\Large
    \begin{tikzpicture}[overlay]
        \draw[blue,fill=blue] (-1.5cm,-6pt) rectangle (\paperwidth, 13pt)
        (\linewidth,16.4pt);
	\end{tikzpicture}}

\usepackage{background}
\backgroundsetup{
	scale=1,	%% Größe (so lassen)
	angle=0,	%% Ausrichtung (Winkel)
	opacity=1,  %% Deckkraft
	contents={\includegraphics[width=\paperwidth,height=\paperheight]{img/blauer_rand.pdf}}
}

\newenvironment{smallitemize}{\begin{itemize}\itemsep -3pt}{\end{itemize}}



\begin{document}

% blauer Rand

\begin{center}
	\LARGE{Betriebsanweisung \enquote{Sprühätzer}}
\end{center}


%\section{\bgvbox{Anwendungsbereich}}
%\section{Anwendungsbereich}
\begin{center}
	Diese Betriebsanweisung fasst die wichtigsten Gefahren und Regeln zusammen.\\
	Für die Bedienung des Sprühätzers ist eine unterschriebene Einweisung für die Platinenfertigung nötig.\\
	Diese Anweisung ist nur zusätzlich zu sehen!\\
\end{center}

\section{Gefahren für Mensch und Umwelt}

\begin{smallitemize}
	\item Benetzung der Haut mit Ätzmittel Eisen(III)-Chlorid.
	\item Sich schneiden, stechen usw. an Werkzeug, Werkstück, Spänen.
	\item Intensiver Hautkontakt mit Kühlschmierstoff kann zu Hautschäden führen.
	\item Gefahr von rotbraunen Flecken auf der Kleidung.
\end{smallitemize}

\section{Schutzmaßnahmen und Verhaltensregeln}

\begin{itemize}
	\item Generell:
	\begin{smallitemize}
		\item Unbedingt Schutzbrille tragen, nach Möglichkeit Handschuhe tragen.
		\item Sprühfunktion am Gerät nur bei geschlossenem Deckel einschalten.
		\item Vor Verwendung Flüssigkeitsstand kontrollieren: Flüssigkeit muss bis zur Unterkante der Trennung zwischen Ätzwanne und Sprühraum stehen.
		\item Bei Flüssigkeitsmangel mit Leitungswasser nachfüllen.
		\item Die Sprühätzung ist deutlich schneller als die Ätzküvette, Ätzergebnis alle 30s kontrollieren.
		
	\end{smallitemize}
	\item Im Betrieb:
	\begin{smallitemize}
		\item Anlage NIE im Betrieb öffnen, Ätzmittel spritzt sofort heraus.
		\item Für Doppelseitige Ätzung unbedingt beide Sprüher einschalten.
		\item An der Platine an zwei gegenüberliegenden Seiten einen 2mm breiten Streifen zur Klemmung freilassen.
		\item 
		
	\end{smallitemize}
	\item Arbeitsanweisung:
	\begin{smallitemize}
		\item 
		\item 
		\item 
		\item 
		\item 
		\item 
		\end{smallitemize}
\end{itemize}

\section{Verhalten bei Störungen und im Gefahrenfall}
\begin{smallitemize}
	\item Bei Schäden oder Störungen an der Maschine: Ausschalten und Betreuer informieren. Schadensmeldung sichtbar an der Maschine anbringen.
	\item evtl. ausgelaufens Ätzmittel mit Papiertüchern aufnehmen und feucht nachwischen.
	\item Schäden nur vom Fachmann beseitigen lassen.
\end{smallitemize}

\section{Verhalten bei Unfällen - Erste Hilfe}
\begin{smallitemize}
	\item Sprühfunktion abschalten. Heizung abschalten.
	\item Ätzmittelspritzer sofort mit viel Wasser abwaschen.
	\item Betreuer informieren. Gegebenenfalls René rufen.
	\item Verletzten betreuen.
\end{smallitemize}

\section{Instandhaltung, Entsorgung}
\begin{smallitemize}
	\item Nach Abschluss der Ätzung Platine unter fließendem Wasser abspühlen.
	\item Deckel wieder auflegen 
		\vspace{5mm}
	\item Für die Instandhaltung ist zuständig:...............................
\end{smallitemize}
\vfill
Copyright:  Diese Seite basiert auf einer Vorlage der BGHM, deshalb ist sie NICHT unter CC-Lizenz veröffentlicht.
\end{document} 